%!TEX TS-program = xelatex

\documentclass {beamer}

\useoutertheme{infolines}

\input {../commonpackages.tex}

%\setmainfont {NanumMyeongjo}
\setsansfont {Noto Sans CJK KR}
\setmainfont {Noto Sans CJK KR}
\setmonofont[Scale=0.8]{DejaVu Sans Mono}

\input {../lststyles.tex}

\hypersetup {
  colorlinks, linkcolor=blue
}

\title {K-공유기 씹고 뜯고 맛보고 즐기기}
\input {../beamauthor.tex}

\AtBeginSection[]
{
  \begin{frame}
    \frametitle{Index}
    \tableofcontents[currentsection]
  \end{frame}
}

\begin {document}

\begin{frame}
  \titlepage
\end{frame}

\section[Section]{K-공유기?}
\begin{frame}
  \frametitle{0x00. Foo?}
  \framesubtitle{Bar!}

  \begin{itemize}
  \item Foo!
  \item<2-> Bar!
  \end{itemize}
\end{frame}

\begin{frame}
  \frametitle{0x00. License}
  \framesubtitle{}
  Copyright (C)  2015 perillamint\linebreak
  Permission is granted to copy, distribute and/or modify this document
  under the terms of the GNU Free Documentation License, Version 1.3
  or any later version published by the Free Software Foundation;\linebreak
  with no Invariant Sections, no Front-Cover Texts, and no Back-Cover Texts.
  A copy of the license is included in the section entitled "GNU
  Free Documentation License".
  \linebreak
  \linebreak
  Repository address:\linebreak
  \url{https://github.com/perillamint/hwhack-101}
  \linebreak
  \linebreak
  \includegraphics [width=30mm]{../img/gfdl-logo-small.png}
\end{frame}

\end {document}
